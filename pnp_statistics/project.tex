\documentclass{article}
\usepackage{amsmath}
\usepackage{amsfonts}
\usepackage{amssymb}
\usepackage{hyperref}
\usepackage[margin=1in]{geometry}

\title{Parametric \& Nonparametric Statistics Project}
\author{Aleksandr Jan Smoliakov}
\date{2024--12--12}

\begin{document}

\maketitle

\section{Introduction}



\section{Preliminaries}

In the project below, we will use the following parameters:
\begin{itemize}
    \item $\mathcal{N} = 9$ (first name: `Aleksandr', 9 letters)
    \item $\mathcal{S} = 9$ (last name: `Smoliakov', 9 letters)
    \item $\mathcal{I}_1 = 5$ (last digit of study book number)
    \item $\mathcal{I}_2 = 8$ (second last digit of study book number)
\end{itemize}

Let \(G_1, \ldots, G_m\) be given distribution functions and \(p_1, \ldots, p_m\) be probabilities that sum to 1. The distribution function \(G\) defined by

\[
G(u) := p_1 G_1(u) + \cdots + p_m G_m(u) = \sum_{k=1}^m p_k G_k(u), \quad u \in \mathbb{R}
\]
is called a mixture of distribution functions \(G_1, \ldots, G_m\) with probabilities (or weights) \(p_1, \ldots, p_m\).

\(G\) is the distribution function of the random variable \(Z\) generated in the following way:

\begin{enumerate}
    \item Choose \(k \in \{1, \ldots, m\}\) at random with probabilities (or weights) \(p_1, \ldots, p_m\). The chosen number is denoted by \(k^*\).
    \item Generate a random variable \(Z^*_{k^*}\) according to the distribution function \(G_{k^*}\) and assign \(Z \leftarrow Z^*_{k^*}\).
\end{enumerate}

In this task, we will have \(m = 2\), so the algorithm for generating \(Z\) is as follows:

\[
Z \leftarrow Z^*_{1+k^*} \quad k^* \sim \text{Binomial}(1, p_2), \quad Z^*_{k} \sim G_k \ (k = 1, 2).
\]

Let
\[
\mathcal{G}(\Theta) = \{G(\cdot|\boldsymbol{\theta}), \boldsymbol{\theta} \in \Theta\}
\]

be a given parametric family of absolutely continuous parametric functions \(G(\cdot|\boldsymbol{\theta})\) with the respective distribution densities \(g(\cdot|\boldsymbol{\theta})\) dependent on the unknown parameter \(\boldsymbol{\theta} \in \Theta\). It is assumed that \(\boldsymbol{\theta}\) is two-dimensional, i.e., \(\boldsymbol{\theta} = (\theta_1, \theta_2) \in \mathbb{R}^2\).

\subsection{Parametric Family Selection}

Using the assigned formula \( \ell := \left\lfloor \frac{\mathcal{I}_2 + 2.5}{2} \right\rfloor \), we find \( \ell = 5 \). Thus, we will use the parametric family \(\mathcal{G}_5(\Theta)\) in this task.

$\mathcal{G}_5(\Theta)$ contains distribution functions of random variables uniformly distributed on $[\theta_1, \theta_2]$, where $\theta_1 < \theta_2$.

It can be expressed as:
\[
G(u|\boldsymbol\theta) = \begin{cases}
0 & u < \theta_1 \\
\frac{u - \theta_1}{\theta_2 - \theta_1} & \theta_1 \le u \le \theta_2 \\
1 & u > \theta_2
\end{cases}
\]
with \(\boldsymbol\theta = (\theta_1, \theta_2) \in \mathbb{R}^2\) and \(\theta_1 < \theta_2\).

\section{Task 1: Testing Goodness-of-Fit}

\subsection{Basic Distribution Function}

The problem gives a specific basic parameter:
\[
\boldsymbol\theta_0 = (-\mathcal{N}, \mathcal{S} + 4) = (-9, 13).
\]

Thus, the basic distribution function is:
\[
G_0(u) = \mathcal{G}_5(u|\boldsymbol\theta_0) = U(-9, 13).
\]

For a uniform distribution \(U(a,b)\):

\begin{itemize}
    \item Mean: \(\mu = \frac{a+b}{2}\)
    \item Variance: \(v^2 = \frac{(b-a)^2}{12}\)
\end{itemize}

For \(G_0 = U(-9, 13)\):

\[
\mu_0 = \frac{-9 + 13}{2} = \frac{4}{2} = 2
\]

\[
v_0^2 = \frac{22^2}{12} = \frac{484}{12} = \frac{121}{3} \approx 40.3333
\]

\subsection{Finding Mixture Distributions}

We are given the following equations for the mixture distributions \(G_1\) and \(G_2\):

\[
\mu_0 = \mu(\boldsymbol\theta_1), \quad \mathcal{N} v_0^2 = v^2(\boldsymbol\theta_1).
\]

\[
\mu_0 + 2v_0 = \mu(\boldsymbol\theta_2), \quad v_0^2 = \mathcal{S} v^2(\boldsymbol\theta_2).
\]

\subsubsection{Determining \(G_1\)}

First we determine \(G_1\). We have:

\[
\mu_0 = \mu(\theta_1), \quad \mathcal{N} v_0^2 = v^2(\theta_1).
\]

It is given that $\mu(\theta_1) = \mu_0 = 2$.

Plugging in \(\mathcal{N} = 9\) and \(v_0^2 = \frac{121}{3} \), we get:
\[
v^2(\boldsymbol\theta_1) = \mathcal{N} v_0^2 = 9 \times \frac{121}{3} = 363.
\]

Let \(G_1(u) = U(a_1, b_1)\). For a uniform distribution:
\[
\mu(\boldsymbol\theta_1) = \frac{a_1 + b_1}{2}, \quad v^2(\boldsymbol\theta_1) = \frac{(b_1 - a_1)^2}{12}.
\]

Since \(\mu(\boldsymbol\theta_1) = 2\):
\[
\frac{a_1 + b_1}{2} = 2 \implies a_1 + b_1 = 4.
\]

Since \(v^2(\boldsymbol\theta_1) = 363\):
\[
\frac{(b_1 - a_1)^2}{12} = 363 \implies b_1 - a_1 = \sqrt{(b_1 - a_1)^2} = \sqrt{4356} = 66.
\]

Solving the system:
\[
a_1 + b_1 = 4, \quad b_1 - a_1 = 66.
\]

Adding the two equations:
\[
2b_1 = 70 \implies b_1 = 35.
\]
\[
a_1 = 4 - 35 = -31.
\]

Thus:
\[
\boldsymbol\theta_1 = (-31, 35) \implies G_1(u) = U(-31, 35).
\]

\subsubsection{Determining \(G_2\)}

Repeating the process for \(G_2\). We have:

\[
\mu_0 + 2v_0 = \mu(\boldsymbol\theta_2), \quad v_0^2 = \mathcal{S} v^2(\boldsymbol\theta_2).
\]

Given \(\mu_0 = 2\) and \(v_0^2 = 40.3333\), we have:

\[
\mu(\boldsymbol\theta_2) = \mu_0 + 2v_0 = 2 + 2 \times \sqrt{40.3333} = 2 + 2 \times 6.3509 = 2 + 12.7018 = 14.7018.
\]

Also:
\[
v_0^2 = \mathcal{S} v^2(\boldsymbol\theta_2) \implies 40.3333 = 9 v^2(\boldsymbol\theta_2) \implies v^2(\boldsymbol\theta_2) = \frac{40.3333}{9} \approx 4.4815.
\]

For \(G_2(u) = U(a_2, b_2)\):
\[
\frac{a_2 + b_2}{2} = 14.7018 \implies a_2 + b_2 = 29.4036.
\]
\[
\frac{(b_2 - a_2)^2}{12} = 4.4815 \implies (b_2 - a_2)^2 = 4.4815 \times 12 = 53.7777.
\]
\[
b_2 - a_2 = \sqrt{53.7777} \approx 7.3333.
\]

Solving the system:
\[
a_2 + b_2 = 29.4036, \quad b_2 - a_2 = 7.3333.
\]

Adding the two equations:
\[
2b_2 = 36.7369 \implies b_2 = 18.3685.
\]
\[
a_2 = 29.4036 - 18.3685 = 11.0351.
\]

Thus:
\[
\boldsymbol\theta_2 = (-11.0351, 18.3685) \implies G_2(u) = U(11.0351, 18.3685).
\]

\subsection{Computing \( p_1 \) and \( p_2 \)}

Given:
\[
\tau = \frac{1}{1+I_1}, \quad I_1 = 5 \implies \tau = \frac{1}{6}.
\]
\[
\alpha_1 = 0.1, \quad \alpha_2 = 0.01.
\]

\[
p_1 = (\alpha_1)^{1-\tau} (\alpha_2)^\tau = (0.1)^{5/6} (0.01)^{1/6} \approx 0.06813.
\]

Then:
\[
p_2 = \frac{5 p_1}{\sqrt{S}} = \frac{5 \times 0.06813}{\sqrt{9}} = \frac{0.3406}{3} \approx 0.1135.
\]

\subsection{Determining Mixture Distributions}

We consider testing:
\[
H_0: F_Y = G_0 \textbf{ versus } H': F_Y \neq G_0
\]

We will compare the empirical distribution of samples generated from:

\begin{enumerate}
    \item \(F_Y = (1-p_1)G_0 + p_1 G_1\), i.e. a mixture of \(G_0\) and \(G_1\).
    \item \(F_Y = (1-p_2)G_0 + p_2 G_2\), i.e. a mixture of \(G_0\) and \(G_2\).
\end{enumerate}

The tests are conducted for sample sizes:
\[
N_1 = 10 \times (2+\mathcal{N}) = 10 \times (2+9) = 110,
\]
\[
N_2 = 100 \times (2+\mathcal{N}) = 100 \times (2+9) = 1100.
\]

\subsection{Goodness-of-Fit Tests}

We will use the Kolmogorov-Smirnov test for the given samples \((Y_t)_{t=1}^n\):

The test statistic is:
\[
D_n = \sup_u |F_n(u) - F(u)|,
\]
where \(F_n\) is the empirical distribution function (EDF) based on the sample and \(F\) is the theoretical distribution function. In this case, \(F = G_0\).

Since:
\[
F_Y(u) = (1 - p_k) G_0(u) + p_k G_k(u),
\]
we have:
\[
F_Y(u) - G_0(u) = p_k [G_k(u) - G_0(u)],
\]
for \(k=1\) or \(k=2\).

Thus, the maximum difference between \(F_Y\) and \(G_0\) is:
\[
\sup_u |F_Y(u)-G_0(u)| = p_k \sup_u |G_k(u)-G_0(u)|.
\]

We need \(\sup_u |G_1(u)-G_0(u)|\) and \(\sup_u |G_2(u)-G_0(u)|\).

\subsubsection{\(G_1\) vs. \(G_0\)}

\(G_0 = U(-9, 13)\), so:
\[
G_0(u)=\begin{cases}
0 & u < -9 \\
\frac{u+9}{22} & -9 \le u \le 13 \\
1 & u > 13
\end{cases}
\]

\(G_1 = U(-31,35)\), so:
\[
G_1(u)=\begin{cases}
0 & u < -31 \\
\frac{u+31}{66} & -31 \le u \le 35 \\
1 & u > 35
\end{cases}
\]

To find \(\sup|G_1(u)-G_0(u)|\), we investigate ranges of \(u\) piecewise between the breakpoints of the two functions.

\begin{enumerate}
    \item For \(u < -31\): \(G_0(u) = G_1(u) = 0\), so the difference is 0.
    \item For \(-31 \le u < -9\): \(G_0(u) = 0\), \(G_1(u) = \frac{u+31}{66}\). The difference is \(\frac{u+31}{66}\), which is increasing as \(u\) approaches -9, where it is \(\frac{-9 + 22}{66} = \frac{1}{3}\).
    \item For \(-9 \le u < 13\): \(G_0(u) = \frac{u+9}{22}\), \(G_1(u) = \frac{u+31}{66}\). The difference is \(\frac{u+31}{66} - \frac{u+9}{22} = \frac{2u-4}{66}\), which is increasing from \(-\frac{1}{3}\) at -9 to \(\frac{1}{3}\) at 13.
    \item For \(13 \le u < 35\): \(G_0(u) = 1\), \(G_1(u) = \frac{u+31}{66}\). The difference is \(\frac{u+31}{66} - 1 = \frac{u-35}{66}\), which is increasing from \(-\frac{1}{3}\) at 13 to 0 at 35.
    \item For \(u \ge 35\): \(G_0(u) = G_1(u) = 1\), so the difference is 0.
\end{enumerate}

The maximum absolute difference is \(\frac{1}{3}\) at the endpoints of the range \([-9,13]\).

Hence:
\[
\sup_u|G_1(u)-G_0(u)| = 1/3 \approx 0.3333.
\]

For the mixture, taking \(p_1 \approx 0.06813\):
\[
\sup_u|F_Y(u)-G_0(u)| = p_1 \times 0.3333 = 0.06813 \times 0.3333 \approx 0.02271.
\]

\subsubsection{\(G_2\) vs. \(G_0\)}

Repeating the process for \(G_2\):

\(G_0 = U(-9, 13)\), so:
\[
G_0(u)=\begin{cases}
0 & u < -9 \\
\frac{u+9}{22} & -9 \le u \le 13 \\
1 & u > 13
\end{cases}
\]

\(G_2 = U(11.0351, 18.3685)\), so:
\[
G_2(u)=\begin{cases}
0 & u < 11.0351 \\
\frac{u-11.0351}{7.3333} & 11.0351 \le u \le 18.3685 \\
1 & u > 18.3685
\end{cases}
\]

To find \(\sup|G_2(u)-G_0(u)|\), we investigate ranges of \(u\) piecewise between the breakpoints of the two functions.

\begin{enumerate}
    \item For \(u < -9\): \(G_0(u) = G_2(u) = 0\), so the difference is 0.
    \item For \(-9 \le u < 11.0351\): \(G_0(u) = \frac{u+9}{22}\), \(G_2(u) = 0\). The difference is \(\frac{u+9}{22}\), which is increasing as \(u\) approaches 11.0351, where it is \(\frac{11.0351 + 9}{22} \approx 0.9107\).
    \item For \(11.0351 \le u < 13\): \(G_0(u) = \frac{u+9}{22}\), \(G_2(u) = \frac{u-11.0351}{7.3333}\). The difference is \(\frac{u-11.0351}{7.3333} - \frac{u+9}{22} = \frac{3u-33.1053}{22} - \frac{u+9}{22} = \frac{2u-42.1053}{22}\), which is increasing from -0.9107 at 11.0351 to \(\frac{13-42.1053}{22} \approx -0.7321\) at 13.
    \item For \(13 \le u < 18.3685\): \(G_0(u) = 1\), \(G_2(u) = \frac{u-11.0351}{7.3333}\). The difference is \(\frac{u-11.0351}{7.3333} - 1 = \frac{u-18.3685}{7.3333}\), which is increasing from \(-\frac{18.3685 - 13}{7.3333} = -\frac{5.3685}{7.3333} \approx -0.7321\) at 13 to 0 at 18.3685.
    \item For \(u \ge 18.3685\): \(G_0(u) = G_2(u) = 1\), so the difference is 0.
\end{enumerate}

The maximum absolute difference is \(\approx 0.9107\) at 11.0351.

Hence:
\[
\sup_u|G_2(u)-G_0(u)| \approx 0.9107.
\]

For the mixture, taking \(p_2 \approx 0.1135\):
\[
\sup_u|F_Y(u)-G_0(u)| = p_2 \times 0.9107 = 0.1135 \times 0.9107 \approx 0.1034.
\]

\subsection{Critical Values and Detection Probability}

Under \(H_0\), the Kolmogorov-Smirnov test critical values at significance \(\alpha_1=0.1\) and \(\alpha_2=0.01\) are approximately:
\[
D_{N,\alpha_1} \approx \frac{1.22}{\sqrt{N}} \quad \text{for} \quad \alpha_1=0.1,
\]
\[
D_{N,\alpha_2} \approx \frac{1.63}{\sqrt{N}} \quad \text{for} \quad \alpha_2=0.01.
\]

For \(N=110\):
\[
D_{110,0.1} \approx \frac{1.22}{\sqrt{110}} \approx 0.1163 \quad \text{and} \quad D_{110,0.01} \approx \frac{1.63}{\sqrt{110}} \approx 0.1554.
\]

\begin{itemize}
    \item For \(G_1\): \(\sup|F_Y-G_0|\approx 0.02271 < 0.1163 < 0.1554\). Thus, at \(N=110\), it's unlikely we reject \(H_0\). \(p > 0.1\)
    \item For \(G_2\): \(\sup|F_Y-G_0|\approx 0.1034 < 0.1163 < 0.1554\). There is some chance to reject at a higher \(\alpha\) level. \(p > 0.1\) (but it's closer to the borderline)
\end{itemize}

For \(N=1100\):
\[
D_{1100,0.1} \approx \frac{1.22}{\sqrt{1100}} \approx 0.03678 \quad \text{and} \quad D_{1100,0.01} \approx \frac{1.63}{\sqrt{1100}} \approx 0.04915.
\]

\begin{itemize}
    \item For \(G_1\): \(\sup|F_Y-G_0|\approx 0.02271 < 0.03678 < 0.04915\). Even with 1100 samples, we will likely not reject \(H_0\) at \(\alpha=0.1\). \(p > 0.1\)
    \item For \(G_2\): \(\sup|F_Y-G_0|\approx 0.03678 < 0.04915 < 0.1034\). We will almost certainly reject \(H_0\) at \(\alpha=0.01\). \(p < 0.01\)
\end{itemize}

Thus, the results of the Kolmogorov-Smirnov test are as follows:

\begin{table}[h]
\centering
\begin{tabular}{|c|c|c|c|}
\hline
\textbf{Mixture} & \textbf{Sample Size} & \textbf{p-value} & \textbf{Result} \\ \hline
\(G_1\) & 110 & \(>0.1\) & No rejection \\ \hline
\(G_1\) & 1100 & \(>0.1\) & No rejection \\ \hline
\(G_2\) & 110 & \(>0.1\) & No rejection \\ \hline
\(G_2\) & 1100 & \(<0.01\) & Rejection at \(\alpha=0.01\) \\ \hline
\end{tabular}
\end{table}

\subsection{Conclusions}

By analytically comparing the theoretical distributions, we have:

\begin{itemize}
    \item \textbf{Mixture with \(G_1\):}\\
    \(\sup|F_Y-G_0|\approx 0.02271\).\\  
    Even at \(N=1100\), we will likely not reject \(H_0\) at \(\alpha=0.1\). The p-value is higher than 0.1.
    \item \textbf{Mixture with \(G_2\):}\\
    \(\sup|F_Y-G_0|\approx 0.1034\).\\
    We will almost certainly reject \(H_0\) at \(\alpha=0.01\) even with 1100 samples. The p-value is \(<0.01\). We will likely not reject with 110 samples at \(\alpha=0.1\), but the p-value may be close.
\end{itemize}

It is evident that for Kolmogorov-Smirnov tests, the magnitude of the deviation from \(G_0\) and the sample size play the decisive role.

As \(N \to \infty\), if \(F_Y \neq G_0\), the empirical distribution \(F_N\) converges to \(F_Y\), and thus \(D_N\) converges to \(\sup_u |F_Y(u)-G_0(u)|\).

\section{Task 2: Applications of Bootstrap Technique}

In this section we will
* test Complex Goodness of Fit Hypothesis,
* check bootstrap consistency,
* compare bootstrap confidence interval construction methods.

We will use the same parametric family \(\mathcal{G}_5(\Theta)\) and the same distributions \(G_0, G_1, G_2\) as in Task 1.

\[
G_0(u) = U(-9, 13), \quad G_1(u) = U(-31, 35), \quad G_2(u) = U(11.0351, 18.3685).
\]

In this section we will use the sample sizes from Task 1, \(N_1 = 110\) and \(N_2 = 1100\).

The bootstrap sample size will be \(B = 100 \times N\), where \(N\) is the original sample size.

\subsection{Testing Goodness-of-Fit by Bootstrap}

First we will test the Complex Goodness of Fit Hypothesis which asserts that the unknown distribution function \(F_Y\) belongs to the parametric family \(\mathcal{G}(\Theta)\):
\[
H_0: F_Y \in \mathcal{G}(\Theta) \textbf{ versus } H': F_Y \notin \mathcal{G}(\Theta).
\]

We will make use of the parametric bootstrap technique to test this hypothesis. The test statistic is the Kolmogorov-Smirnov test statistic and the significance levels are \(\alpha = 0.1\).



\subsection{Checking Bootstrap Consistency}

\subsection{Bootstrap Confidence Intervals}

\end{document}
