\documentclass[svgnames, 12pt]{beamer}

\usepackage[utf8]{inputenc}
\usepackage[english]{babel}
\usepackage[L7x]{fontenc}
\usepackage{lmodern}
\usepackage{amsmath}
\usepackage{amssymb}
\usepackage{xcolor}
\usepackage{subfig}
\usepackage{graphicx}
\usepackage{lipsum}
\usepackage{hyperref}

\definecolor{mifcolor}{RGB}{0, 71, 127}
\definecolor{dimgr}{RGB}{105, 105, 105}
\definecolor{sky}{RGB}{0, 191, 255}
\setbeamercolor{alerted text}{fg=red,bg=sky}
\newcommand{\boxalert}[1]{{%
	\usebeamercolor{alerted text}\colorbox{bg}{\alert{#1}}%
}}

\mode<presentation>{
\usetheme{Madrid}
\usecolortheme[named=mifcolor]{structure}
\setbeamertemplate{footline}
{%
	\leavevmode%
	\hbox{ \begin{beamercolorbox}[wd=.3\paperwidth,ht=2.5ex,dp=1.125ex,leftskip=.3cm
			plus1fill,rightskip=.3cm]{author in head/foot}%
			\usebeamerfont{author in head/foot}\insertshortauthor \hfill 
		\end{beamercolorbox}%
		\begin{beamercolorbox}[wd=.2\paperwidth,ht=2.5ex,dp=1.125ex,leftskip=.3cm,
			rightskip=.3cm plus1fil]{institute in head/foot}%
			\usebeamerfont{institute in head/foot}\insertshortinstitute
		\end{beamercolorbox}%
		\begin{beamercolorbox}[wd=.2\paperwidth,ht=2.5ex,dp=1.125ex,leftskip=.3cm,
			rightskip=.3cm plus1fil]{date in head/foot}%
			\usebeamerfont{date in head/foot}\insertshortdate
		\end{beamercolorbox}%
		\begin{beamercolorbox}[wd=.3\paperwidth,ht=2.5ex,dp=1.125ex,leftskip=.3cm,
			rightskip=.3cm plus1fil]{title in head/foot}%
			\usebeamerfont{title in head/foot}\insertshorttitle\hfill p.
			\insertframenumber\enspace of \inserttotalframenumber\enspace 
	\end{beamercolorbox} }%
	\vskip0pt%
}
}

\title[FDA of Weather Data]{Functional Data Analysis of Weather Data}%
\author[A. J. Smoliakov]{Aleksandr Jan Smoliakov\inst{1}}
\institute[VU MIF]{\inst{1} Vilnius University, Faculty of Mathematics and Informatics}
\date{2025--03--18}

\begin{document}

\begin{frame}
\includegraphics[scale=0.15]{MIF Garamond-logo.png} 
\hfill
\includegraphics[scale=0.15]{Logo_spalvotas.eps}

\titlepage
\end{frame}

\begin{frame}{Table of Contents}
\tableofcontents
\end{frame}

\section{Data Source}

\begin{frame}{Data Source}
			\begin{itemize}
		\item \textbf{Historical Weather Data in India}
		\begin{itemize}
			\item Hourly observations (2006--2019)
			\item 8 major Indian cities
			\item Over 20 meteorological variables (focus on Temperature)
		\end{itemize}
			\end{itemize}
\end{frame}

\section{Research Questions}

\begin{frame}{Research Questions}
	\begin{itemize}
		\item \textbf{Main goal:} Explore daily patterns in temperature data.
		\item \textbf{Questions:}
		\begin{itemize}
			\item How do daily temperature curves differ by season?
			\item Are there consistent patterns in the derivatives of the curves?
		\end{itemize}
	\end{itemize}
\end{frame}

\section{Temperature Curves}

\begin{frame}{Unsmoothed Temperature Curves}
	\begin{itemize}
		\item Daily temperature profiles from Mumbai in 2011.
			\item Lines colored by season (winter, spring, summer, fall).
			\item "Stepwise" appearance due to hourly observations.
		\end{itemize}
	\begin{center}
		\includegraphics[width=0.8\linewidth]{../notebooks/assets/unsmoothed_temperature_curves.png}
	\end{center}
\end{frame}

\begin{frame}{Lambda Search: Smoothing Parameter Tuning}
    \begin{itemize}
        \item Looking for the optimal smoothing parameter \(\lambda\).
		\item GCV evaluated for \(\lambda\) values from \(10^{-4}\) to \(10^{0.6}\).
    \end{itemize}
    \vfill
    \begin{center}
        \includegraphics[width=0.7\linewidth]{../notebooks/assets/lambda_vs_gcv.png}
    \end{center}
\end{frame}

\begin{frame}{Smoothed Temperature Curves}
		\begin{itemize}
		\item Applied B-spline smoothing (8 basis functions, $\lambda=0.5$).
		\item Smoothed curves reveal underlying daily patterns.
	\end{itemize}
	\begin{center}
		\includegraphics[width=0.8\linewidth]{../notebooks/assets/smoothed_temperature_curves.png}
	\end{center}
\end{frame}

\section{Derivatives \& Covariance}

\begin{frame}{Derivative Analysis}
	\begin{itemize}
		\item Plotted first derivative (slope) and second derivative (acceleration) of the smoothed curves.
		\item Noticeable consistency in the daily patterns.
		\end{itemize}
	\begin{center}
		\includegraphics[width=0.48\linewidth]{../notebooks/assets/slope_fd_plot.png}\hfill
		\includegraphics[width=0.48\linewidth]{../notebooks/assets/acceleration_fd_plot.png}
	\end{center}
\end{frame}

\begin{frame}{Covariance Analysis}
		\begin{itemize}
		\item Estimated the covariance function from the smoothed functional data.
		\item Interesting valleys indicate periods of low covariance.
	\end{itemize}
	\begin{center}
		\includegraphics[width=0.6\linewidth]{../notebooks/assets/covariance_heatmap.png}
	\end{center}
\end{frame}

\section{FPCA}

\begin{frame}{Functional Principal Component Analysis}
	\begin{itemize}
		\item Performed FPCA on the smoothed temperature curves.
		\item First 3 PC explain over 98\% of the variance.
	\end{itemize}
	\begin{center}
		\includegraphics[width=0.8\linewidth]{../notebooks/assets/pca_scree_plot.png}
	\end{center}
\end{frame}

\section{Summary \& Future Work}

\begin{frame}{Summary \& Future Work}
	\begin{itemize}
		\item \textbf{Key findings:}
		\begin{itemize}
			\item FDA can reveal daily patterns in temperature data.
		\end{itemize}
		\item \textbf{Next steps:}
		\begin{itemize}
			\item Decide on the more specific direction of the analysis.
			\item Incorporate yearly time series data.
			\item Use Fourier basis functions for smoothing.
			\item Dig deeper with more advanced FDA techniques.
		\end{itemize}
	\end{itemize}
\end{frame}

\begin{frame}{Thank You!}
	\begin{center}
		\Huge Thank you for your attention!
	\end{center}
\end{frame}

\end{document}
